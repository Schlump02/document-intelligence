\section{Demo-Seite}
Auf dieser Seite "[\ldots] befinden sich Umsetzungsbeispiele für häufig benötigte Elemente im Fließtext [...]"\directcite{theisen2011}, so die Vorlage w\"ortlich. \\

\begin{center}
    \begin{table}[h]
    \centering
    \begin{tabular}{|c|p{6cm}|}
        \hline
        \textbf{Datum} & \textbf{Aktivitäten} \\
        \hline
        Kebab & 7€ \\
        \hline
        Adana & \begin{itemize}
            \item \textbf{Groß}: 8€
            \item \textbf{Klein}: 6€
        \end{itemize} \\
        \hline
        Köfte & \begin{itemize}
            \item 5 Stück: 8€
            \item 2 Stück: 6€
        \end{itemize}\\
        \hline
        Mercimek Suppe & 3€ \\
        \hline
        Dönerteller & 15€ \\
        \hline
    \end{tabular}
    \captionwithfootnotemark{Beispiel Tabelle.}% Die (verpflichtende) Quellenangabe hierzu wird durch den \footcitetext-Befehl weiter unten gesetzt
    \label{tab:example}
    \end{table}
\end{center}
\footcitetext[Vgl.][\printfield{pages}]{DemoQuelle}\\ [-4em]

Die Tabelle zeigt den Preis eines Dönertellers, dieser lässt sich wie folgt berechnen:
\begin{equation}
    15 = \sum_{n=1}^{10} \frac{n}{20} + \sum_{k=1}^{5} \frac{2k}{10} - \sum_{i=1}^{3} i
\end{equation}

\newpage
Der folgende Abschnitt könnte hilfreich für eine Ausarbeitung in der Informatik sein.
\begin{figure}[h]
    \begin{lstlisting}
        def hello_world():
            print("Hello, World!")
    \end{lstlisting}
    \captionwithfootnotemark{Ausschnitt aus main.py.}% Die (verpflichtende) Quellenangabe hierzu wird durch den \footnotetext-Befehl weiter unten gesetzt
    \label{fig:meincode}
\end{figure}
\footnotetext{Quelle: Eigene Erstellung}

\image[0.6]{deckbild.jpeg}{logo}{Das Logo der BA}{theisen2011}{6}

\subsection{Abkürzungen aus dem Verzeichnis}

Diese \ac{CPU} verarbeitet Daten in einem Takt von mehreren \ac{kHz}.\\
Das kleine Kind war ganz fasziniert von den \acp{BluM}, und wollte unbedingt eine eigene \ac{BluM} als Spielzeug haben. 
Auf dem Kinderfest waren zahlreiche \acp{BluM} zu sehen, jede bunter und fröhlicher als die andere.\\

Erzwungene Kurzschreibweise: \acs{CPU}\\
Erzwungene Langschreibweise: \acl{CPU}

\newpage
\subsection{Zitierbeispiele}
\subsubsection{Beispiel für jeden Quelltyp}% "Vgl." steht nur in Fußnoten für indirekte (nicht wörtliche) Zitate.
% Wenn eine Quellenangabe sich auf einen ganzen Satz bzw. Absatz bezieht, wird die Fußnote erst am Satzende nach dem Punkt platziert.

Buch/Monografie\indirectcite{theisen2011}\\
Sammelwerk\indirectcite{maier2004}\\
Zeitschriften-/Journalartikel\indirectcite{chodorowreich2022loan}\\
Zeitungsartikel\indirectcite{dick2012neugierige}\\
Internet\indirectcite[ ]{capital2014}\\% falls es keine Seitenzahlen gibt, ein Leerzeichen als Seitenzahl-Argument übergeben (wird nicht in der PDF eingetragen)
Gesetzestext\footnote{Vgl. §433 Abs. 1 Satz 1 BGB}\nocite{bgb}\\
Gerichtsurteil\indirectcite{bverfgh1968}\\
öffentliches Dokument\indirectcite{eu2022access}\\
internes Dokument\indirectcite{abcorganigramm}\\% muss auf separatem Datenträger beigefügt werden. Am besten sammelt man schon frühzeitig alle Dokumente in einem Ordner.
(unvollständige Quellenangaben)\indirectcite{blankmaier}

\newpage
\subsubsection{Beispiele für verschiedene Arten von Zitaten}% Es genügt den Anforderungen, Kurzbelege zu setzen, die auf den Vollbeleg im Literaturverzeichnis verweisen.

indirektes Zitat (Seitenzahlen aus der .bib)\indirectcite{theisen2011}\\% Indirekte (sinngemäße) Zitate geben den Inhalt der Quelle in eigenen Worten wieder.
indirektes Zitat (individuelle Seitenzahlen)\indirectcite[23f.]{theisen2011}\\% f. (folgend) bezeichnet diese und die nächste Seite
direktes Zitat (Seitenzahlen aus der .bib)\directcite{theisen2011}\\% Direkte Zitate geben den Wortlaut der Quelle wieder. Sie müssen in Anführungszeichen gesetzt werden.
direktes Zitat (individuelle Seitenzahlen)\directcite[23f.]{theisen2011}

\subsubsubsection{Beispiel für eine sub-sub-sub-Überschrift}
Diese Vorlage unterstützt (technisch und moralisch) nur max. vierfache Untergliederung.

\subsection{Displayquote}
\begin{displayquote}
    Barrierefrei sind bauliche und sonstige Anlagen, Verkehrsmittel, technische Gebrauchsgegenstände, Systeme der Informationsverarbeitung, akustische und visuelle Informationsquellen und Kommunikationseinrichtungen sowie andere gestaltete Lebensbereiche, wenn sie für Menschen mit Behinderungen in der allgemein üblichen Weise, ohne besondere Erschwernis und grundsätzlich ohne fremde Hilfe auffindbar, zugänglich und nutzbar sind. Hierbei ist die Nutzung behinderungsbedingt notwendiger Hilfsmittel zulässig.\footnote{§ 4 BGG}
\end{displayquote}

\iffalse
\subsection{Listen}
\begin{enumerate}
    \item First item
    \item Second item
    \item Third item
    \item Fourth item
    \item Certainly! Here is an example of the modified LaTeX code with Lorem Ipsum text added to the items: item
    \item Sixth item
    \item Seventh item
    \item Eighth item
    \begin{enumerate}
        \item Second level item 1
        \item Second level item 2
        \item Second level item 3
        \item Second level item 4
        \item Second level item 5
        \item Second level item 6
        \item Second level item 7Certainly! Here is an example of the modified LaTeX code with Lorem Ipsum text added to the items: item
        \begin{center}
            \begin{table}[h]
            \centering
            \begin{tabular}{|c|p{6cm}|}
                \hline
                \textbf{Datum} & \textbf{Aktivitäten} \\
                \hline
                Kebab & 7€ \\
                \hline
                Adana & \begin{itemize}
                    \item \textbf{Groß}: 8€
                    \item \textbf{Klein}: 6€
                \end{itemize} \\
                \hline
                Köfte & \begin{itemize}
                    \item 5 Stück: 8€
                    \item 2 Stück: 6€
                \end{itemize}\\
                \hline
                Mercimek Suppe & 3€ \\
                \hline
                Dönerteller & 15€ \\
                \hline
            \end{tabular}
            \captionwithfootnotemark{Beispiel Tabelle.}% Die (verpflichtende) Quellenangabe hierzu wird durch den \footcitetext-Befehl weiter unten gesetzt
            \label{tab:example}
            \end{table}
        \end{center}
        \footcitetext[Vgl.][\printfield{pages}]{DemoQuelle}\\ [-4em]
        \item Second level item 8Certainly! Here is an example of the modified LaTeX code with Lorem Ipsum text added to the items: item
        \item Second level item 9
        \item Second level item 10Certainly! Here is an example of the modified LaTeX code with Lorem Ipsum text added to the items: item
        \item Second level item 11
        \item Second level item 12
        \begin{enumerate}
            \renewcommand{\theenumii}{\theenumi.\arabic{enumii}}
            \item Third level item
            \item Third level itemCertainly! Here is an example of the modified LaTeX code with Lorem Ipsum text added to the items: item
            \item Third level item
            \begin{enumerate}
                \renewcommand{\theenumii}{\theenumi.\arabic{enumii}}
                \item Fourth level item
                \item Fourth level itemCertainly! Here is an example of the modified LaTeX code with Lorem Ipsum text added to the items: item
                \item Fourth level item
            \end{enumerate}
        \end{enumerate}
    \end{enumerate}
    \item Ninth item
    \item Tenth item
    \item Eleventh item
    \item Twelfth item
    \item Thirteenth item
    \item Fourteenth item
    \item Fifteenth item
    \item Eleventh item
    \item Twelfth item
    \item Thirteenth item
    \item Fourteenth item
    \item Fifteenth item
\end{enumerate}
hi
\begin{itemize}
    \item First item
    \item Second item
    \begin{itemize}
        \item unordered second level item 1
        \item unordered second level item 2 Certainly! Here is an example of the modified LaTeX code with Lorem Ipsum text
        added to the items: item
        \item unordered second level item 3
        \begin{itemize}
            \item unordered Third level item 1
            \item unordered Third level item 2 Certainly! Here is an example of the modified LaTeX code with Lorem Ipsum text
            added to the items: item
            \item unordered Third level item 3
            \begin{itemize}
                \item unordered Fourth level item 1
                \item unordered Fourth level item 2 Certainly! Here is an example of the modified LaTeX code with Lorem Ipsum text
                added to the items: item
                \item unordered Fourth level item 3
            \end{itemize}
        \end{itemize}
    \end{itemize}
    \item Third item
\end{itemize}
\fi